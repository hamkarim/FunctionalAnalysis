\documentclass{unswmaths}

\usepackage{unswshortcuts}

\begin{document}

\subject{Functional Analysis}
\author{Edward McDonald}
\title{Assignment 2}
\studentno{3375335}


\newcommand{\Real}{\operatorname{Re}}
\newcommand{\Img}{\operatorname{Im}}
\newcommand{\lan}{\langle}
\newcommand{\ran}{\rangle}

\setlength\parindent{0pt}

\unswtitle{}

\section*{Question 1}
\begin{theorem}
    Suppose $(V,\lan,\ran)$ is a complex inner product space. We assume that $\lan,\ran$
    is linear in the first argument. Then
    if $T \in L(V)$, and for all $x \in V$, $\lan x,Tx\ran \geq 0$. Then $T = T^*$.
\end{theorem}
\begin{proof}
    Let $x,y \in V$. Then,
    \begin{equation*}
        \lan x+iy,T(x+iy)\ran \geq 0.
    \end{equation*}
    Expanding this out, we find
    \begin{equation*}
        \lan x,Tx\ra + \lan y,Ty\ra + i\lan y,Tx\ran-i\lan x,Ty\ran \geq 0. 
    \end{equation*}
    Hence the number
    \begin{equation*}
        i(\lan y,Tx\ran-\lan x,Ty\ran)
    \end{equation*}
    is real, so choose $r \in \Rl$ such that
    \begin{equation}
    \label{re1}
        \lan y,Tx\ran -\lan x,Ty\ran = ir.
    \end{equation}
    So take conjugates,
    \begin{equation}
    \label{re2}
        \lan Tx,y\ran-\lan Ty,x\ran = -ir.
    \end{equation}
    Then add equation \ref{re1} to equation \ref{re2} to find
    \begin{align*}
        \lan y,Tx\ra - \lan Ty,x\ran &= \lan x,Ty\ran-\lan Tx,y\ra\\
        &= -\overline{\lan y,Tx\ran - \lan Ty,x\ran}.
    \end{align*}
    So therefore,
    \begin{equation*}
        \Real(\lan y,Tx\ran - \lan Ty,x\ran) = 0.
    \end{equation*}
    
    Now put $z = iy$. Since $x$ and $y$ were arbitrary, we then have
    \begin{equation*}
        \Real(\lan z,Tx\ran - \lan Tz,x\ran) = 0.
    \end{equation*}
    So,
    \begin{equation*}
        \Real[i(\lan y,Tx\ran-\lan Ty,x\ran)] = 0.
    \end{equation*}
    Thus,
    \begin{equation*}
        \Img(\lan y,Tx\ran-\lan Ty,x\ran) = 0.
    \end{equation*}
    Hence $\lan y,Tx\ran = \lan Ty,x\ra$ for all $x,y \in V$. So $T = T^*$.
\end{proof}


\section*{Question 2}
\subsection*{Part 1}
\begin{lemma}
\label{shiftSum}
    Suppose that the sequence of complex numbers $\{a_n\}_{n\in\mathbb{Z}}$ is absolutely summable. That is,
    \begin{equation*}
        \sum_{n\in \mathbb{Z}} |a_n| \leq \infty.
    \end{equation*}
    Then the operator
    \begin{equation*}
        T = \sum_{n\geq0} a_n S^n + \sum_{n> 0} a_{-n} (S^*)^n
    \end{equation*}
    exists and is a bounded linear operator
    on $\ell^2(\mathbb{N})$, where $S$ is the forward shift operator on $\ell^2(\mathbb{N})$.
\end{lemma}
\begin{proof}
    Since $\|S\| = \| S^*\| = 1$, the series
    \begin{equation*}
        \sum_{n\geq0} \|a_n S^n\| + \sum_{n> 0} \|a_{-n} (S^*)^n\|
    \end{equation*}
    is finite. Hence by the completeness of $\mathcal{B}(\ell^2(\mathbb{N}))$, the operator
    $T$ exists and is bounded.
\end{proof}
\begin{proposition}
    If $\{a_n\}_{n\in \mathbb{Z}}$ is an absolutely summable sequence as in lemma
    \ref{shiftSum}. Then the infinite matrix $(a_{j-k})_{j,k\geq 0}$ defines
    a bounded linear operator on $\mathcal{B}(\ell^2(\mathbb{N}))$.
\end{proposition}
\begin{proof}
    Let $T_A$ be the (possibly not everywhere defined) operator given by the matrix
    $(a_{j-k})_{j,k\geq 0}$. Then, 
    \begin{equation*}
        T_A = \sum_{n\geq0} a_n S^n + \sum_{n> 0} a_{-n} (S^*)^n
    \end{equation*}
    hence $T_A$ is bounded by lemma \ref{shiftSum}.
\end{equation*}
\end{proof}
\subsection*{Part 2}
\begin{lemma}
    Let $\Hilb = L^2(0,1)$, and $\varphi \in L^\infty(0,1)$. Let $M_\varphi$ be the pointwise
    multiplication operator by $\varphi$ defined on $\Hilb$.
\end{lemma}


\end{document}
