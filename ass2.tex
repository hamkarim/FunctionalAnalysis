\documentclass{unswmaths}

\usepackage{unswshortcuts}

\begin{document}

\subject{Functional Analysis}
\author{Edward McDonald}
\title{Assignment 2}
\studentno{3375335}


\newcommand{\Real}{\operatorname{Re}}
\newcommand{\Img}{\operatorname{Im}}
\newcommand{\lan}{\langle}
\newcommand{\ran}{\rangle}
\newcommand{\Proj}{\mathbb{P}_+}
\newcommand{\isom}{\cong}


\unswtitle{}

\section*{Question 1}
\begin{theorem}
    Suppose $(V,\lan,\ran)$ is a complex inner product space. We assume that $\lan,\ran$
    is linear in the first argument. Then
    if $T \in L(V)$, and for all $x \in V$, $\lan x,Tx\ran \geq 0$. Then $T = T^*$.
\end{theorem}
\begin{proof}
    Let $x,y \in V$. Then,
    \begin{equation*}
        \lan x+iy,T(x+iy)\ran \geq 0.
    \end{equation*}
    Expanding this out, we find
    \begin{equation*}
        \lan x,Tx\ra + \lan y,Ty\ra + i\lan y,Tx\ran-i\lan x,Ty\ran \geq 0. 
    \end{equation*}
    Hence the number
    \begin{equation*}
        i(\lan y,Tx\ran-\lan x,Ty\ran)
    \end{equation*}
    is real, so choose $r \in \Rl$ such that
    \begin{equation}
    \label{re1}
        \lan y,Tx\ran -\lan x,Ty\ran = ir.
    \end{equation}
    So take conjugates,
    \begin{equation}
    \label{re2}
        \lan Tx,y\ran-\lan Ty,x\ran = -ir.
    \end{equation}
    Then add equation \ref{re1} to equation \ref{re2} to find
    \begin{align*}
        \lan y,Tx\ra - \lan Ty,x\ran &= \lan x,Ty\ran-\lan Tx,y\ra\\
        &= -\overline{\lan y,Tx\ran - \lan Ty,x\ran}.
    \end{align*}
    So therefore,
    \begin{equation*}
        \Real(\lan y,Tx\ran - \lan Ty,x\ran) = 0.
    \end{equation*}
    
    Now put $z = iy$. Since $x$ and $y$ were arbitrary, we then have
    \begin{equation*}
        \Real(\lan z,Tx\ran - \lan Tz,x\ran) = 0.
    \end{equation*}
    So,
    \begin{equation*}
        \Real[i(\lan y,Tx\ran-\lan Ty,x\ran)] = 0.
    \end{equation*}
    Thus,
    \begin{equation*}
        \Img(\lan y,Tx\ran-\lan Ty,x\ran) = 0.
    \end{equation*}
    Hence $\lan y,Tx\ran = \lan Ty,x\ra$ for all $x,y \in V$. So $T = T^*$.
\end{proof}


\section*{Question 2}
\subsection*{Part 1}
\begin{lemma}
\label{shiftSum}
    Suppose that the sequence of complex numbers $\{a_n\}_{n\in\mathbb{Z}}$ is absolutely summable. That is,
    \begin{equation*}
        \sum_{n\in \mathbb{Z}} |a_n| \leq \infty.
    \end{equation*}
    Then the operator
    \begin{equation*}
        T = \sum_{n\geq0} a_n S^n + \sum_{n> 0} a_{-n} (S^*)^n
    \end{equation*}
    exists and is a bounded linear operator
    on $\ell^2(\mathbb{N})$, where $S$ is the forward shift operator on $\ell^2(\mathbb{N})$.
\end{lemma}
\begin{proof}
    Since $\|S\| = \| S^*\| = 1$, the series
    \begin{equation*}
        \sum_{n\geq0} \|a_n S^n\| + \sum_{n> 0} \|a_{-n} (S^*)^n\|
    \end{equation*}
    is finite. Hence by the completeness of $\mathcal{B}(\ell^2(\mathbb{N}))$, the operator
    $T$ exists and is bounded.
\end{proof}
\begin{proposition}
    If $\{a_n\}_{n\in \mathbb{Z}}$ is an absolutely summable sequence as in lemma
    \ref{shiftSum}. Then the infinite matrix $(a_{j-k})_{j,k\geq 0}$ defines
    a bounded linear operator on $\mathcal{B}(\ell^2(\mathbb{N}))$.
\end{proposition}
\begin{proof}
    Let $T_A$ be the (possibly not everywhere defined) operator given by the matrix
    $(a_{j-k})_{j,k\geq 0}$. Then, 
    \begin{equation*}
        T_A = \sum_{n\geq0} a_n S^n + \sum_{n> 0} a_{-n} (S^*)^n
    \end{equation*}
    hence $T_A$ is bounded by lemma \ref{shiftSum}.
\end{proof}
\subsection*{Part 2}
\begin{definition}
    Let $e_n(t) = exp(2\pi i n t)$, and if $\varphi \in C[0,1]$,
    and the multiplication operator $M_\varphi:L^2(0,1)\rightarrow L^2(0,1)$ is defined by $M_\varphi f(t) = \varphi(t)f(t)$
    for all $t$ in a dense subset of $(0,1)$.    
    
    Denote the fourier coefficients of $f \in L^2(0,1)$ by $\hat{f}(n) = \lan f,e_n\ran$.
\end{definition}
\begin{definition}
    The Riesz projection, $\mathbb{P}_+:L^2(0,1)\rightarrow L^2(0,1)$ is the linear map given by
    \begin{equation*}
        \mathbb{P}_+\sum_{n\in\mathbb{Z}} a_ne_n = \sum_{n\geq 0} a_ne_n.
    \end{equation*}
    This is well defined since $\{e_n\}_{n\in mathbb{Z}}$ is a complete orthonormal basis.
\end{definition}
\begin{lemma}
    The Riesz projection is bounded, and an orthogonal projection.
\end{lemma}
\begin{proof}
    By the Parseval identity,
    \begin{align*}
        \| \Proj \sum_{n\mathbb{Z}} a_n e_n\|^2 &= \sum_{n\geq 0} |a_n|^2\\
        &\leq \sum_{n\in\mathbb{Z}}|a_n|^2\\
        &= \|\sum_{n\in\mathbb{Z}} a_ne_n\|^2.
    \end{align*}
    Clearly $\Proj^2 = \Proj$, and 
    \begin{equation*}
        \lan e_n,\Proj e_k\ran = \begin{cases}
            1\text{ if k }\geq 0\text{ and }n = k\\
            0\text{otherwise}
        \end{cases}
    \end{equation*}
    which is identical to $\lan\Proj e_n,e_k\ran$. Hence $\Proj = \Proj^*$.    
\end{proof}
\begin{definition}
    Since $\Proj$ is an orthogonal projection, $\Proj L^2(0,1)$ is a closed
    subspace, hence a Hilbert space. Let $\Hlbt = \Proj L^2(0,1)$. 
\end{definition}
\begin{remark}
    $\Hlbt$ has basis $\{e_n\}_{n\geq 0}$ and $\Hlbt \isom \ell^2(\mathbb{N})$
    since the map $n\mapsto e_n$ is an isometric isomorphism.
\end{remark}
\begin{lemma}
\label{gammaMatrix}
    Let $\varphi \in C[0,1]$. Then the operator $\Gamma_\varphi := \Proj M_\varphi :\Hlbt\rightarrow\Hlbt$
    is a bounded linear operator, represented in the basis $\{e_n\}_{n\geq 0}$
    of $\Hlbt$ by the infinite Toeplitz matrix $(\hat{\varphi(k-n)})_{n,k\geq 0}$.
\end{lemma}
\begin{proof}
    Since $\varphi$ is continuous on $[0,1]$, then $M_\varphi$ is bounded. Hence since $\Proj$
    is bounded, $\Gamma_\varphi$ is bounded. 
    
    Now for $n,k\geq 0$,
    \begin{align*}
        \lan \Gamma_\varphi e_n,e_k\ran &= \sum_{j\geq0} \lan \hat{\varphi(j)}e_{n+j},e_k\ran\\
        &= \sum_{j\geq 0} \hat{\varphi(j)} \lan e_{n+j},e_k\ran\\
        &= \hat{\varphi(k-n)}.
    \end{align*}
    
    Hence, if $f \in \Hlbt$, then
    \begin{align*}
        \hat{\Gamma_\varphi f}(n) = \sum_{j\geq 0} \hat{\varphi(n-j)}\hat{f}(j).
    \end{align*}
    So $\Gamma_\varphi$ is given by the required infinite Toeplitz matrix.
\end{proof}
\begin{theorem}
    There is a Toeplitz operator on $\ell^2$ with infinite matrix $(a_{n-k})_{n,k\geq 0}$
    where the sum $\sum_{n\in \mathbb{Z}} |a_n|$ does not converge.
\end{theorem}
\begin{proof}
    By lemma \ref{gammaMatrix}, it is sufficient to find $\varphi \in C[0,1]$
    where the sum $\sum_{n\in \mathbb{Z}} |\hat{\varphi}(n)|$
    does not converge.
    Consider
    \begin{equation*}
        \varphi(t) = t.
    \end{equation*}
    Then
    \begin{equation*}
        \hat{\varphi}(n) = \begin{cases}
            -\frac{1}{2\pi i n}\text{ if }n\neq 0\\
            0\text{ if }n = 0.
        \end{cases}
    \end{equation*}
\end{proof}

\begin{theorem}
    The only Hilbert-Schmidt operator that is a Toeplitz operator
    is $0$.
\end{theorem}
\begin{proof}
    Suppose that $T$ is a Toeplitz operator. So $T$ has an infinite matrix
    representation, $(a_{j-k})_{j,k\geq 0}$. If $T$ is Hilbert Schmidt, then
    \begin{equation*}
        \sum_{j,k\geq 0} |a_{j-k}|^2 < \infty.
    \end{equation*}
    However this sum contains infinitely many terms of $|a_n|^2$ for each $n$.
    Hence the sum converges if and only if $a_n = 0$ for all $n$.
    
    Hence $T = 0$.
\end{proof} 
\section*{Question 3}
For this question, $T$ is a linear operator
on a Hilbert space $\Hlbt$.
\begin{proposition}
\label{3(1)}
    $(\im T)^\perp = \ker T^*$
\end{proposition}
\begin{proof}
    Suppose $x \in \ker T^*$. Then for all $y \in \Hlbt$,
    \begin{align*}
        0 &= \lan T^*x,y\ran\\
          &= \lan x,Ty\ran.
    \end{align*}
    Hence $x \in (\im T)^\perp$. 
    
    Now suppose $x \in (\im T)^\perp$. Hence for all
    $y \in \Hlbt$, 
    \begin{align*}
        0 &= \lan Ty,Tx\ran\\
        &= \lan y,T^*x\ran.
    \end{align*}
    Now put $y = T^*x$, and so $\| T^*x\| = 0$. Hence
    $x \in \ker T^*$.
\end{proof}
\begin{proposition}
    $(\ker T)^\perp = \overline{\im T^*}$.
\end{proposition}
\begin{proof}
    We have from proposition \ref{3(1)} that for
    any operator $S$, $(\im S)^\perp = \ker S^*$.
    
    Put $S = T^*$. Then
    \begin{equation*}
        (\im T^*)^\perp = \ker T
    \end{equation*}
    Now take the perpendicular complement of both
    sides,
    \begin{equation*}
        \overline{\im T^*} = (\ker T)^\perp
    \end{equation*}
    as required.
\end{proof}

\end{document}
