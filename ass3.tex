\documentclass{unswmaths}

\usepackage{unswshortcuts}

\begin{document}

\subject{Functional Analysis}
\author{Edward McDonald}
\title{Assignment 3}
\studentno{3375335}


\newcommand{\Real}{\operatorname{Re}}
\newcommand{\Img}{\operatorname{Im}}
\newcommand{\lan}{\langle}
\newcommand{\ran}{\rangle}
\newcommand{\Proj}{\mathbb{P}_+}
\newcommand{\isom}{\cong}
\newcommand{\id}{{\operatorname{id}}}


\unswtitle{}
\section*{Question 1}
For this question, $X$ is a Banach space and $X_0$ is a closed subspace.

\begin{proposition}
    There is an isometric embedding, $X/X_0 \hookrightarrow (X_0^\perp)^*$.
\end{proposition}
\begin{proof}
    Define $\rho:X/X_0\rightarrow (X_0^\perp)^*$ as $\rho(x+X_0)(f) = f(x)$, for $x \in X$
    and $f \in X_0^\perp$. This is a well defined linear map, since if $x' + X_0 = x+X_0$, then 
    $f(x) = f(x')$ since $x-x' \in X_0$ and $f \in X_0^\perp$. $\rho$ is linear, since $\rho(\alpha x+y+X_0) = f(\alpha x+y) = \alpha f(x) + f(y)$
    for $x,y \in X$ and $\alpha \in \Cplx$.
    
    
    We need to show that $\rho$ is an isometry. That is,
    \begin{equation*}
        \| \rho(x+X_0)\|_{(X_0^\perp)^*} = \| x+X_0\|_{X/X_0}.
    \end{equation*}
    for any $x \in X$.
    
    Expanding this out into definitions, we must prove that
    \begin{equation}
    \label{isometryCondition}
        \sup_{\|f\|_{X^*} \leq 1, f \in X_0^\perp} |f(x)| = \inf_{y \in X_0} \|x-y\|.
    \end{equation}
    for any $x \in X$.
    
    Suppose that $f \in X_0^\perp$ with $\|f\|_{X^*} \leq 1$, then for any $y \in X_0$
    and $x \in X$.
    \begin{equation*}
        |f(x)| = |f(x-y)| \leq \|x-y\|.
    \end{equation*}
    Hence,
    \begin{equation*}
        |f(x)| \leq \inf_{y \in X_0} \|x-y\|.
    \end{equation*}
    So
    \begin{equation*}
        \sup_{\|f\|_{X^*} \leq 1, f \in X_0^\perp} |f(x)| \leq \inf_{y \in X_0} \|x-y\|
    \end{equation*}
    follows. Now we must prove the opposite inequality. 
    
    Let $x \in X$. Then on the subspace $V := \Cplx x$, define the functional
    \begin{equation*}
        \omega(\lambda x) = \lambda \|x+X_0\|_{X/X_0}.
    \end{equation*}
    $\omega$ is linear, and for any $y \in V$, $|\omega(y)| = \|y+X_0\| \leq \|y\|$.  So $\|\omega\| \leq 1$.
    
    So $\omega$ is a linear functional on a subspace $V$ of $X$ bounded by the seminorm $\|\cdot\|_{X/X_0}$. 
    
    So by the Hahn-Banach theorem, there is a functional $f \in X^*$, with $f(y) = \omega(y)$ for $y \in V$
    and $|f(z)| \leq \|z+X_0\|$ for any $z \in X$. Hence for $z \in X_0$, $f(z) = 0$, so $f \in X_0^\perp$
    and $|f(z)|\leq \|z\|$, so $\|f\|_{X^*}\leq 1$. 
    
    Therefore, $|f(x)| = \|x+X_0\|_{X/X_0}$. Hence, 
    \begin{equation*}
        \sup_{\|f\|_{X^*} \leq 1, f \in X_0^\perp} |f(x)| \geq \|x+X_0\|_{X/X_0} =  \inf_{y \in X_0} \|x-y\|.
    \end{equation*}
    
    So the equality \ref{isometryCondition} holds. Hence $\rho$ is an isometric embedding.    
\end{proof}

%\begin{lemma}
%    There is an isometric isomorphism,
%    \begin{equation*}
%        X_0^\perp \isom (X/X_0)^*
%    \end{equation*}
%\end{lemma}
%\begin{proof}
%    Define $\rho:X_0^\perp \rightarrow (X/X_0)^*$ as follows:
%    \begin{equation*}
%        \rho(f)(x+X_0) = f(x).
%    \end{equation*}
%    For $f \in X_0^\perp$ and $x + X_0 \in X/X_0$. This is well defined,
%    since if we chose a different coset representative, $x' + X_0 = x+X_0$, then
%    $f(x) = f(x')$ since $X_0 \subset \ker f$ as $f \in X_0^\perp$.
    
%    $\rho$ is clearly linear, we need only prove that it is an isometry. That is,
%    we must prove
%    \begin{equation*}
%        \| \rho(f) \|_{(X/X_0)^*} = \|f\|_{X^*}.
%    \end{equation*}
%    Or if write the definitions of these norms, we must prove
%    \begin{equation*}
%        \sup_{\|x+X_0\|\leq 1} |f(x)| = \sup_{\|x\|\leq 1} |f(x)|.
%    \end{equation*}
%    
%    Since $\|x+X_0\| \leq \|x\|$, it is clear that
%    \begin{equation*}
%        \sup_{\|x+X_0\|\leq 1} |f(x)| \geq \sup_{\|x\|\leq 1} |f(x)|.
%    \end{equation*}
%    So we must prove the reverse inequality.
%    
%    Suppose $x \in X$ with $\|x+X_0\| < 1$. Then there is a sequence $\{y_n\}_{n=1}^\infty \subset X_0$
%    such that $\|x+y_n\| \rightarrow \|x+X_0\|$ and $\|x+y_n\| < 1$. Then $|f(x)| = |f(x+y_n)|$,
%    there is a point $x'$ in the set $\{ x\; \|x\| \leq 1\}$ such that $|f(x')| = |f(x)|$. 
%    So we can conclude
%    \begin{equation*}
%                \sup_{\|x+X_0\| < 1} |f(x)| \leq \sup_{\|x\|\leq 1} |f(x)|.%
%    \end{equation*}
%    Note the strict inequality on the left hand side.
%    
%    Now consider $x \in X$ with $\|x+X_0\| = 1$. Then there is a sequence $x_n+X_0$
%    such that $x_n + X_0\rightarrow x+X_0$ with $\|x_n+X_0\| < 1$. Then $|f(x_n)| \rightarrow |f(x)|$
%    and so we conclude 
%    \begin{equation*}
%                \sup_{\|x+X_0\| \leq 1} |f(x)| \leq \sup_{\|x\|\leq 1} |f(x)|.
%    \end{equation*}
    
%    Hence $\rho$ is an isometric embedding.
%    
%    To show that $\rho$ is surjective, we consider
%    \begin{equation*}
%        \pi: (X/X_0)^*\rightarrow X_0^\perp
%    \end{equation*}
%    given by $\pi(f)(x) = f(x+X_0)$ for $f \in (X/X_0)^*$ and $x \in X$.
%    
%    See that $\pi\circ\rho = \id_{X_0^\perp}$ and $\rho\circ\pi = \id_{(X/X_0)^*}$. 
    
%    Hence $\rho$ has a inverse, so is an isometric isomorphism.    
%\end{proof}

%\begin{theorem}
%    There is an isometric embedding,
%    \begin{equation*}
%        X/X_0 \hookrightarrow (X_0^\perp)^*
%    \end{equation*}
%\end{theorem}
%\begin{proof}
%    Since $X_0^\perp \isom (X/X_0)^*$ is an isometric isomorphism, we have $(X/X_0)^{**} \isom (X_0^\perp)^*$. 
%    Hence since $X/X_0\hookrightarrow (X/X_0)^{**}$ isometrically,
%    the result follows.
%\end{proof}
\section*{Question 2}
\begin{theorem}
    Suppose $T:X\rightarrow Y$ is a linear mapping between normed spaces $X$ and $Y$. Then $T$ is bounded
    if and only if $T$ has the property if that if $U$ is open in $Y$ then $T^{-1}(U)$ is open in $X$.
\end{theorem}
\begin{proof}
    Suppose that $T$ is bounded, and let $U\subset Y$ be an open set with $T^{-1}(U) \neq \emptyset$. 
    
    Then let $x \in T^{-1}(U)$, and let $\varepsilon$ be small enough such that $B_Y(Tx,\varepsilon) \subset U$. 
    
    Then choose $\varepsilon' = \varepsilon/\|T\|$. Then if $y \in B_X(x, \varepsilon')$, 
    \begin{equation*}
        \|Ty-Tx\| \leq \|T\|\|y-x\| \leq \varepsilon.
    \end{equation*}
    Hence $Ty \in B_Y(Tx,\varepsilon)$, so $Ty \in U$. Therefore $B_X(x,\varepsilon') \subset T^{-1}(U)$, and so $T^{-1}(U)$
    is open.
    
    Conversely, suppose that $T$ has the property that $T^{-1}(U)$ is open in $X$ whenever $U$ is open in $Y$.
    
    Let $U = B_Y(0,1) \subset Y$. Then since $T^{-1}(U)$ is open, there is some $\varepsilon > 0$ such
    that $B_X(0,\varepsilon) \subset T^{-1}(U)$.
    
    So $TB_X(0,\varepsilon) \subseteq B_Y(0,1)$. Hence, by linearity, $TB_X(0,1) \subseteq B_Y(0,\frac{1}{\varepsilon})$.
    
    Hence, $\|T\| \leq 1/\varepsilon$. So $T$ is bounded.
    
\end{proof}

\section*{Question 3}
Let $X = \ell^1(\mathbb{N})$, and $X_0$ is the subspace defined by
\begin{equation*}
    X_0 = \{ (\xi_k)_{k\geq 0} \in X \;:\;\sum_{k\geq 0} \xi_k = 0\}.
\end{equation*}
\begin{theorem}
    There is an isometric isomorphism,
    \begin{equation*}
        X/X_0 \isom \Cplx
    \end{equation*}
\end{theorem}
%\begin{proof}
%    Let $S:X\rightarrow \Cplx$ be the function given by
%    \begin{equation*}
%        S((\xi)_{k\geq 0}) = \sum_{k\geq 0} \xi_k.
%    \end{equation*}
%    $S$ is linear since for $(\xi_k)_{k\geq 0},(\xi'_k)_{k\geq 0} \in X$, and $\alpha \in \Cplx$
%    \begin{equation*}
%        S(\alpha(\xi_k)_{k\geq 0}+(\xi'_k)_{k\geq 0}) = \sum_{k\geq 0} (\alpha\xi_k+\xi'_k) = \alpha\sum_{k\geq 0}\xi_k + \sum_{k\geq 0}\xi'_k = \alphs S((\xi_k)_{k\geq 0}) + S((\eta_k)_{k\geq 0}).
%    \end{equation*}
%    since we may rearrange the sum as it is absolutely convergent.
%    
%    $S$ has image $\Cplx$, since for any $\alpha \in \Cplx$, $S(\alpha,0,0,\ldots) = \alpha$.
%    
%    Hence by the first isomorphism theorem, we have $X/X_0 \isom \Cplx$ as complex vector spaces.
%    
%    We now show that this is an isometry. Suppose
%\end{proof}
\begin{proof}
    Define the function $S:X\rightarrow \Cplx$ by
    \begin{equation*}
        S((\xi_k)_{k\geq 0}) = \sum_{k\geq 0} \xi_k.
    \end{equation*}
    This is well defined, since if $(\xi')_{k\geq 0} \in X_0$, then 
%    \begin{equation*}
%        S((\xi_k)_{k\geq 0}+(\xi'_k)_{k\geq 0}) = \sum_{k\geq 0} \xi_k+\xi'_k = \sum_{k\geq 0}\xi_k + \sum_{k\geq 0}\xi'_k = S((\xi_k)_{k\geq 0}),
%    \end{equation*}
%    

%    $S$ is linear, since if $(\eta_k)_{k\geq0},(\xi_k)_{k\geq0} \in X$ and $\alpha \in \Cplx, then
%    \begin{equation*}
%        S((\xi_k)_{k\geq 0}+(\eta_k)_{k\geq 0}) = \sum_{k\geq 0} (\xi_k+\eta_k) = \sum_{k\geq 0}\xi_k + \sum_{k\geq 0} \eta_k
%    \end{equation*}
%    since we may rearrange the sum as it is absolutely convergent. 

    $S$ is clearly linear and $S$ is surjective since $S(\lambda,0,0,0,\ldots) = \lambda$ for any $\lambda \in \Cplx$.
    
    So by the first isomorphism theorem, there is a vector space isomorphism,
    \begin{equation*}
        X/\ker S = X/X_0 \isom \Cplx.
    \end{equation*}
    
    Let $\Psi$ be this isomorphism, that is,
    $\Psi(x+X_0) = S(x)$ for $x \in X$.
    
    Now we must show that this is an isometry.
    
    Choose $(x_k)_{k\geq 0} \in X$ such that each $x_k \geq 0$
    and $\sum_{k\geq 0} x_k = S((x_k)_{k\geq 0}) = 1$. For example, $x_k = \frac{1}{2^{k+1}}$.
    
    See that $\|(x_k)_{k\geq 0}+X_0\|_{X/X_0} = 1$, since 
    
    
    Since we have shown $X/X_0$ is one dimensional, $X/X_0 = \Cplx((x_k)_{k\geq 0}+X_0)$. Hence, any $y + X_0 \in X/X_0$
    is of the form $y = \lambda((x_k)_{k\geq 0}+X_0)$ for some $\lambda \in \Cplx$. Then,
    \begin{equation*}
        |\Psi(y)| = |S(\lambda((x_k)_{k\geq 0})| = |\lambda| = \|y\|_{X/X_0}.
    \end{equation*}
    So $\Psi$ is an isometry.
\end{proof}
\begin{theorem}
    $X_0^\perp$ is the one dimensional subspace of $\ell^\infty$, $\Cplx(1,1,1,\ldots)$, where $(1,1,1,\ldots)$
    is a constant sequence with value $1$.
\end{theorem}
\begin{proof}
    We identify $X^*$ with $\ell^\infty$. For $(\eta_k)_{k\geq 0} \in X_0^\perp$, we require for all $(\xi_k)_{k\geq 0} \in X_0$,
    \begin{equation*}
        \sum_{k\geq 0} \xi_k \overline{\eta_k} = 0.
    \end{equation*}
    Choose the sequence $(\xi_k)_{k\geq 0} \in X_0$ as $\xi_0 = 1$ and $\xi_p = -1$ for some $p > 0$,
    and $\xi_k = 0$ otherwise. Then,
    \begin{equation*}
        \eta_0 - \eta_p = 0.
    \end{equation*}
    Hence $(\eta_k)_{k\geq 0}$ is a constant sequence. So $X_0^\perp$ consists of constant sequences. Clearly
    any constant sequence is in $X_0^\perp$, so $X_0^\perp = \Cplx(1,1,1,\ldots)$.
\end{proof}

\end{document}
