\documentclass{unswmaths}

\usepackage{unswshortcuts}

\begin{document}

\subject{Functional Analysis}
\author{Edward McDonald}
\title{Assignment 3}
\studentno{3375335}


\newcommand{\Real}{\operatorname{Re}}
\newcommand{\Img}{\operatorname{Im}}
\newcommand{\lan}{\langle}
\newcommand{\ran}{\rangle}
\newcommand{\Proj}{\mathbb{P}_+}
\newcommand{\isom}{\cong}
\newcommand{\id}{{\operatorname{id}}}


\unswtitle{}
\section*{Question 1}
For this question, $X$ is a Banach space and $X_0$ is a closed subspace.
\begin{lemma}
    There is an isometric isomorphism,
    \begin{equation*}
        X_0^\perp \isom (X/X_0)^*
    \end{equation*}
\end{lemma}
\begin{proof}
    Define $\rho:X_0^\perp \rightarrow (X/X_0)^*$ as follows:
    \begin{equation*}
        \rho(f)(x+X_0) = f(x).
    \end{equation*}
    For $f \in X_0^\perp$ and $x + X_0 \in X/X_0$. This is well defined,
    since if we chose a different coset representative, $x' + X_0 = x+X_0$, then
    $f(x) = f(x')$ since $X_0 \subset \ker f$ as $f \in X_0^\perp$.
    
    $\rho$ is clearly linear, we need only prove that it is an isometry. That is,
    we must prove
    \begin{equation*}
        \| \rho(f) \|_{(X/X_0)^*} = \|f\|_{X^*}.
    \end{equation*}
    Or if write the definitions of these norms, we must prove
    \begin{equation*}
        \sup_{\|x+X_0\|\leq 1} |f(x)| = \sup_{\|x\|\leq 1} |f(x)|.
    \end{equation*}
    
    Since $\|x+X_0\| \leq \|x\|$, it is clear that
    \begin{equation*}
        \sup_{\|x+X_0\|\leq 1} |f(x)| \geq \sup_{\|x\|\leq 1} |f(x)|.
    \end{equation*}
    So we must prove the reverse inequality.
    
    Suppose $x \in X$ with $\|x+X_0\| < 1$. Then there is a sequence $\{y_n\}_{n=1}^\infty \subset X_0$
    such that $\|x+y_n\| \rightarrow \|x+X_0\|$ and $\|x+y_n\| < 1$. Then $|f(x)| = |f(x+y_n)|$,
    there is a point $x'$ in the set $\{ x\; \|x\| \leq 1\}$ such that $|f(x')| = |f(x)|$. 
    So we can conclude
    \begin{equation*}
                \sup_{\|x+X_0\| < 1} |f(x)| \leq \sup_{\|x\|\leq 1} |f(x)|.
    \end{equation*}
    Note the strict inequality on the left hand side.
    
    Now consider $x \in X$ with $\|x+X_0\| = 1$. Then there is a sequence $x_n+X_0$
    such that $x_n + X_0\rightarrow x+X_0$ with $\|x_n+X_0\| < 1$. Then $|f(x_n)| \rightarrow |f(x)|$
    and so we conclude 
    \begin{equation*}
                \sup_{\|x+X_0\| \leq 1} |f(x)| \leq \sup_{\|x\|\leq 1} |f(x)|.
    \end{equation*}
    
    Hence $\rho$ is an isometric embedding.
    
    To show that $\rho$ is surjective, we consider
    \begin{equation*}
        \pi: (X/X_0)^*\rightarrow X_0^\perp
    \end{equation*}
    given by $\pi(f)(x) = f(x+X_0)$ for $f \in (X/X_0)^*$ and $x \in X$.
    
    See that $\pi\circ\rho = \id_{X_0^\perp}$ and $\rho\circ\pi = \id_{(X/X_0)^*}$. 
    
    Hence $\rho$ has a inverse, so is an isometric isomorphism.    
\end{proof}

\begin{theorem}
    There is an isometric embedding,
    \begin{equation*}
        X/X_0 \hookrightarrow (X_0^\perp)^*
    \end{equation*}
\end{theorem}
\begin{proof}
    Since $X_0^\perp \isom (X/X_0)^*$ is an isometric isomorphism, we have $(X/X_0)^{**} \isom (X_0^\perp)^*$. 
    Hence since $X/X_0\hookrightarrow (X/X_0)^{**}$ isometrically,
    the result follows.
\end{proof}
\section*{Question 2}
\begin{theorem}
    Suppose $T:X\rightarrow Y$ is a linear mapping between normed spaces $X$ and $Y$. Then $T$ is bounded
    if and only if $T$ has the property if that if $U$ is open in $Y$ then $T^{-1}(U)$ is open in $X$.
\end{theorem}
\begin{proof}
    Suppose that $T$ is bounded, and let $U\subset Y$ be an open set with $T^{-1}(U) \neq \emptyset$. 
    
    Then let $x \in T^{-1}(U)$, and let $\epsilon$ be small enough such that $B(Tx,\epsilon) \subset U$. 
    
    Then choose $\epsilon' = \epsilon/\|T\|$. Then if $y \in B(x, \epsilon')$, 
    \begin{equation*}
        \|Ty-Tx\| \leq \|T\|\|y-x\| \leq \epsilon.
    \end{equation*}
    Hence $Ty \in B(Tx,\epsilon)$, so $Ty \in U$. Therefore $B(x,\epsilon') \subset T^{-1}(U)$, and so $T^{-1}(U)$
    is open.
    
    Conversely, suppose that $T$ has the property that $T^{-1}(U)$ is open in $X$ whenever $U$ is open in $Y$.
    
    Let $U = B(0,1) \subset Y$. Then since $T^{-1}(U)$ is open, there is some $\epsilon > 0$ such
    that $B(0,\epsilon) \subset T^{-1}(U)$. Hence, $\|T\| \leq 1/\epsilon$. So $T$ is bounded.
    
\end{proof}

\section*{Question 3}
Let $X = \ell^1(\mathbb{N})$, and $X_0$ is the subspace defined by
\begin{equation*}
    X_0 = \{ (\xi_k)_{k\geq 0} \;:\;\sum_{k\geq 0} \xi_k = 0\}.
\end{equation*}
\begin{theorem}
    There is an isometric isomorphism,
    \begin{equation*}
        X/X_0 \isom \Cplx
    \end{equation*}
\end{theorem}
\begin{proof}
    Define function $S:X/X_0\rightarrow \Cplx$ by
    \begin{equation*}
        S((\xi_k)_{k\geq 0}+X_0) = \sum_{k\geq 0} \xi_k.
    \end{equation*}
    This is well defined, since if $(\xi')_{k\geq 0} \in X_0$, then 
    \begin{equation*}
        S((\xi_k)_{k\geq 0}+(\xi'_k)_{k\geq 0}) = \sum_{k\geq 0} \xi_k+\xi'_k = \sum_{k\geq 0}\xi_k + \sum_{k\geq 0}\xi'_k = S((\xi_k)_{k\geq 0}),
    \end{equation*}
    since we may rearrange the sum as it is absolutely convergent. $S$ is linear, since if $(\eta_k)_{k\geq0},(\xi_k)_{k\geq0} \in X$, then
    \begin{equation*}
        S((\xi_k)_{k\geq 0}+(\eta_k)_{k\geq 0}) = \sum_{k\geq 0} \xi_k+\eta_k = \sum_{k\geq 0}\xi_k + \sum_{k\geq 0} \eta_k.
    \end{equation*}
    
    Furthermore, $S$ is bijective. $S$ is surjective since $S(\lambda,0,0,0,\ldots) = \lambda$ for any $\lambda \in \Cplx$,
    and if $S(x+X_0) = S(y+X_0)$, then $x-y \in X_0$, so $x+X_0 = y+Y_0$. 
    
    
    
     
    
\end{proof}


\end{document}
