\documentclass[10pt]{article}
\usepackage{amsmath,amssymb,graphicx,color}
\title{Functional Analysis Assignment 1}
\author{Edward McDonald z3375335}
\date{}

\newtheorem{theorem}{Theorem}
\newtheorem{lemma}[theorem]{Lemma}
\newtheorem{proposition}[theorem]{Proposition}
\newtheorem{corollary}[theorem]{Corollary}
\newenvironment{proof}[1][Proof]{\begin{trivlist}
\item[\hskip \labelsep {\bfseries #1}]}{\end{trivlist}}
\newenvironment{definition}[1][Definition]{\begin{trivlist}
\item[\hskip \labelsep {\bfseries #1}]}{\end{trivlist}}
\newenvironment{example}[1][Example]{\begin{trivlist}
\item[\hskip \labelsep {\bfseries #1}]}{\end{trivlist}}
\newenvironment{remark}[1][Remark]{\begin{trivlist}
\item[\hskip \labelsep {\bfseries #1}]}{\end{trivlist}}

\newcommand{\qed}{\nobreak \ifvmode \relax \else
	      \ifdim\lastskip<1.5em \hskip-\lastskip
		        \hskip1.5em plus0em minus0.5em \fi \nobreak
				      \vrule height0.75em width0.5em depth0.25em\fi}


\topmargin=-10mm
\textheight=250mm
\oddsidemargin=-4mm
\textwidth=166mm

\parskip=2pt
\parindent=0pt

\setlength{\parindent}{0pt}
\usepackage{fullpage}
\usepackage{enumerate}
\newcommand{\im}{\operatorname{im}}
\newcommand{\isom}{\cong}
\newcommand{\modulo}[1]{\;\operatorname{mod} #1}
\newcommand{\Char}{\operatorname{char}}
\newcommand{\tr}{\operatorname{tr}}
\newcommand{\dist}{\operatorname{dist}_{L^\infty}}

\newcommand{\legendre}[2]{\left(\frac{#1}{#2}\right)}

\newcommand{\Span}{\operatorname{span}}

\begin{document}
    \maketitle{}
    \section*{Question 1}
    \begin{theorem}
        Suppose that $B_k$ is the $k$th Bernoulli polynomial,
        and $e_n(t) = \exp(2\pi i n t)$. Then for $n \neq 0$, we have
        \begin{equation*}
            \langle B_k,e_n\rangle = -\frac{k!}{(2\pi i n)^k}
        \end{equation*}
        and
        \begin{equation*}
            \langle B_k,e_0\rangle = 0.
        \end{equation*}
    \end{theorem}
    \begin{proof}
        Put $n\neq 0$.
    
        Consider first $k = 1$. By definition,
        $B_1(t) = t-\frac{1}{2}$. Then,
        \begin{align*}
            \langle B_1,e_n \rangle &= \int_{0}^1\! (x-\frac{1}{2})e^{-2\pi inx}dx\\
            &= \int_0^1\!xe^{-2\pi i nx}dx
        \end{align*}
        since $\int_0^1e^{2\pi inx}dx = 0$. Then using integration by parts,
        \begin{equation*}
            \int_{0}^1\!xe^{-2\pi inx}dx = -\frac{1}{2\pi i n}.
        \end{equation*}
        Hence the result holds for $k = 1$. Suppose now that $k > 1$.
        
        The $k$th Bernoulli polynomial can be found by the fomula $kB_{k-1}(t) = B'_k(t)$
        when $k > 1$.
        Since $B_k(0) = B_k(1)$, and $e_n(0) = e_n(1)$, we have the relation
        \begin{equation*}
            \langle B_k',e_n\rangle = -\int_{0}^1 B_k(t)\frac{d}{dt}[e^{-2\pi i n t}]dt
        \end{equation*}
        This gives us
        \begin{equation*}
            2\pi i n\langle B_k,e_n\rangle = k\langle B_{k-1},e_n\rangle.
        \end{equation*}
        Thus we have a recurrence relation:
        \begin{equation*}
            \langle B_k,e_n\rangle = \frac{k}{2\pi i n}\langle B_{k-1},e_n\rangle
        \end{equation*}
        valid for $k > 1$. Hence by induction,
        \begin{equation*}
            \langle B_k,e_n\rangle = \frac{k!}{(2\pi in)^{k-1}}\langle B_1,e_n\rangle.
        \end{equation*}
        So finally,
        \begin{equation*}
            \langle B_k,e_n \rangle = -\frac{k!}{(2\pi in)^k}
        \end{equation*}
        for $n\neq 0$.
        
        For $n = 0$, we have
        \begin{align*}
            \langle B_k,e_0\rangle &= \int_0^1\!B_k(x)dx\\
            &= \int_0^1\frac{1}{k+1}B'_{k+1}(x)dx\\
            &= \frac{1}{k+1}(B_{k+1}(1)-B_{k+1}(0))\\
            &= 0
        \end{align*} 
        as required.
        $\Box$
    \end{proof} 
    Hence we have the expansion,
    \begin{equation*}
        B_k(t) = \sum_{n\in \mathbb{Z}\setminus\{0\}} -\frac{k!}{(2\pi i n)^k}e^{2\pi i nt}.
    \end{equation*}
    
    \section*{Question 2}
    For this question, $t$ is a variable defined on $[-1,1]$, and we consider
    functions in $L^2(-1,1)$.
        \begin{definition}
            The $n$th Legendre polynomials $L_n$ is
            \begin{equation*}
                L_n(t) := \frac{1}{2^n n!}\sqrt{\frac{2n+1}{2}}\frac{d^n}{dx^n}[(t^2-1)^n]
            \end{equation*}
            and $L_0(t) := 1$.
            Define $v_n(t) = t^n$.
        \end{definition}
        
        \begin{lemma}
            \label{innerProduct}
            \begin{equation*}
                \langle L_n,v_m\rangle  = \begin{cases}
                    0\;\text{ if }n>m\\
                    0\;\text{ if }m-n\text{ is odd and positive.}\\
                    \frac{1}{2^n} \sqrt{\frac{2n+1}{2}}{m\choose n}B(\frac{m-n+1}{2},n+1)\;\text{ if }m-n\text{ is even and positive.}
                \end{cases}
            \end{equation*}
               where $B$ is the beta function.
        \end{lemma}
        \begin{proof}
            Since the function $f(t) = t^2-1$ has $f(0) = f(1) = 0$, we can use the integration
            by parts rule,
            \begin{align*}
                \langle L_n,v_m\rangle &= \frac{1}{2^nn!}\sqrt{\frac{2n+1}{2}}\int_{-1}^1\frac{d^n}{dx^n}[(x^2-1)^n]x^mdx\\
                &= \frac{(-1)^n}{2^n n!}\sqrt{\frac{2n+1}{2}}\int_{-1}^1(x^2-1)^n\frac{d^n}{dx^n}(x^m)dx.
            \end{align*}
            If $n > m$, $\frac{d^n}{dx^n}x^m$ vanishes, and we get $\langle L_n,v_m\rangle = 0$. 
            
            In the case $n \leq m$, we have $\frac{d^n}{dx^n}(x^m) = \frac{m!}{(m-n)!}x^{m-n}$. Hence,
            \begin{equation*}
                \langle L_n,v_m\rangle = \frac{1}{2^n}\sqrt{\frac{2n+1}{2}}{m\choose n} \int_{-1}^1(1-x^2)^nx^{m-n}dx.
            \end{equation*}
            Now if $n-m$ is odd, this is an integral of an odd function over a symmetric
            domain. So $\langle L_n,v_m\rangle = 0$ when $n-m$ is odd.
            
            Now if $n-m$ is even,  use the substitution $u = x^2$ so that
            \begin{equation*}
                \langle L_n,v_m\rangle = \frac{1}{2^n}{m\choose n}\sqrt{\frac{2n+1}{2}}\int_{0}^1(1-u)^nu^{\frac{m-n}{2}-\frac{1}{2}}du.
            \end{equation*}
            Hence we have by the definition of the beta function,
            \begin{equation*}
                \langle L_n,v_m\rangle = \frac{1}{2^n}{m\choose n}\sqrt{\frac{2n+1}{2}}B(\frac{m-n+1}{2},n+1)
            \end{equation*}
            as required. $\Box$
        \end{proof}
        
        \begin{theorem}
            The set $\{L_n\}_{n=1}^\infty$, when normalised, is the result of Gram-Schmidt orthonormalisation
            applied to the set $\{v_n\}_{n=0}^\infty$. 
        \end{theorem}
        \begin{proof}
            Define the set $\{B_n\}_{n=0}^\infty$ as the result of Gram-Schmidt orthogonalisation
            applied to $\{v_n\}_{n=0}^\infty$. That is,
            \begin{align*}
                B_0 &= v_0\\
                B_n &= v_n-\sum_{k=0}^{n-1} \frac{\langle B_k,v_n\rangle}{\langle B_k,B_k\rangle} B_k.
            \end{align*}
            
            Write $P_n = \Span\{v_k\}_{k=0}^n$ as the set of
            polynomials of degree at most $n$. Note that by construction $\Span\{B_k\}_{k=0}^n = P_n$.
            
            We show that $B_n$ is parallel $L_n$ for $n\geq 0$. For $n = 0$, we
            have $B_0(t) = 1 = L_0(t)$.
                        
            By construction, $B_n$ is in the orthogonal complement of $P_{n-1}$
            in $P_n$. Since $P_{n-1}$ is $n$ dimensional, and $P_n$ is $n+1$
            dimensional, the orthogonal complement of $P_{n-1}$
            in $P_n$ is one dimensional.
            
            Hence to show that $L_n$ is parallel
            to $B_n$ it will be enough to show that $L_n$ lies
            in the orthogonal complement to $P_{n-1}$ in $P_n$, since then
            both $B_n$ and $L_n$ lie in the same one dimensional subspace of $P_n$.
            
            Clearly since $L_n$ is a polynomial of degree $n$, we have $L_n\in P_n$. 
            
            By lemma \ref{innerProduct}, $\rangle L_n,v_k\langle = 0$ for $k < n$.
            Hence $L_n$ is orthogonal to every element $P_{n-1}$. 
            
            
            
            
            %, and assume that $B_k$ is parallel to $L_k$ for $k < n$. Then we
            %have
            %\begin{equation*}
            %    B_n = v_n-\sum_{k=0}^{n-1}\langle L_k,v_n\rangle L_k
            %\end{equation*}
            %By construction, $\langle B_n,L_k\rangle = 0$ for $k < n$. 
            %For $k \geq n$, we have
            %\begin{equation*}
            %    \langle B_n,L_k\rangle  = \langle v_n,L_k\rangle.
            %\end{equation*}
            %since the $L_k$ are orthonormal. However by lemma \ref{innerProduct}, 
            %the right hand side is zero for $k > n$, and nonzero
            %for $k = n$.
            
            %That is, $\langle L_k, B_n\rangle$ is nonzero only when $k = n$.
            %Since the $L_n$ form a complete orthonormal system,
            %we therefore have that $B_n$ is parallel to $L_n$.
            
            Hence, $L_n$ is parallel to $B_n$. So if we normalise
            each $L_n$, we obtain the result of Gram-Schmidt orthonormalisation
            applied to the set $\{v_n\}_{n=0}^\infty$. $\Box$
        \end{proof}
        
        
        
        \section*{Question 3}
        
        Lemma \ref{innerProduct} gives us the decomposition of $v_n$ in terms of $L_n$. So,
        \begin{align*}
            v_m &= \sum_{n=0}^\infty \langle L_n,v_m\rangle L_n\\
            &= \sum_{n=0}^m\langle L_n,v_m\rangle L_n
        \end{align*}
        since $\langle L_n,v_m\rangle$ vanishes for $n>m$.
        So if $m$ is even, we have
        \begin{equation*}
            v_m = \sum_{k=0}^{m/2}\frac{1}{2^{2k}} {m\choose 2k}\sqrt{\frac{4k+1}{2}}B(\frac{m-2k+1}{2},n+1)L_{2k}.
        \end{equation*}
        If $m$ is odd, we have
        \begin{equation*}
            v_m = \sum_{k=0}^{(m-1)/2} \frac{1}{2^{2k+1}}{m\choose 2k+1}\sqrt{\frac{4k+2}{2}}B(\frac{m-2k-1}{2},n+1)L_{2k+1}.
        \end{equation*}
\end{document}
